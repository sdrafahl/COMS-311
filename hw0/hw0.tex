\documentclass[11pt]{article}
    %	options include 12pt or 11pt or 10pt
    %	classes include article, report, book, letter, thesis

    \usepackage{amsmath}
    \usepackage{array}
    \setlength\extrarowheight{2pt}
    \usepackage{graphicx}
    \graphicspath{ {/home/shanedrafahl/coms331/hw0} }

    \title{HW1}
    \author{Shane Drafahl}
    \date{30 August,2017}

    \begin{document}
    \maketitle

    1. Use mathematical induction to prove the following statements
    $ \newline \newline $
    (a)
    $ \newline $
    $ \forall_{n} \geq 1 $, $ \sum_{i = 1}^{n} i^{3} $ = $ ( \sum_{i = 1}^{n} i )^{2} $
    $ \newline \newline $
    $ \newline $
    lemma: $ \sum_{i = 1}^{n} i $ = $ \frac{n^{2}}{2} $ + $ \frac{n}{2} $ where $ n \in N $
    $ \newline $
    Base Case: $ \sum_{i = 1}^{1} i $ = 1 = $ \frac{1^{2}}{2} $ + $ \frac{1}{2} $
    $ \newline $
    Inductive Hypothesis: $ \sum_{i = 1}^{n} i $ = $ \frac{n^{2}}{2} $ + $ \frac{n}{2} $
    $ \newline $
    Inductive Steps:
    $ \newline $
    Using induction on n
    $ \newline $
    $ \frac{(n + 1)^{2}}{2} + \frac{n + 1}{2} $ = $ \frac{n^{2}}{2} $ + $ \frac{3n}{2} $ + 1 =
    $ \frac{n^{2}}{2} + \frac{n}{2} $ + $ (n + 1) $
    $ \newline \newline $
    So therefore the lemma is true for all natural numbers n greater than 1.
    $ \newline \newline $
    With the lemma we can solve the proof easier
    $ \newline \newline $
    Base Case: For n = 1 would be $ \sum_{i = 1}^{1} i^{3} $ = $ 1^{3} $ = 1.
    $ ( \sum_{i = 1}^{1} i )^{2} $ = 1.
    $ \newline \newline $
    Inductive Hypothesis: From the base n is true for
    $ \forall_{n} \geq 1 $, $ \sum_{i = 1}^{n} i^{3} $ = $ ( \sum_{i = 1}^{n} i )^{2} $
    $ \newline \newline $
    Inductive Step: Using lemma we can re-write this to
    $ (\frac{n^{2}}{2}  +  \frac{n}{2})^{2} $ = $ \sum_{i = 1}^{n} i^{3} $
    We will first add n+1, $( \frac{(n + 1)^{2}}{2}  +  \frac{(n + 1)}{2})^{2} $ = $ \frac{n^{4}}{4} $ +
    $ \frac{3n^{3}}{2} $ + $ \frac{13n^{2}}{4} $ + 3n + 1 .
    Now we will do $ ( \frac{n^{2}}{2}  +  \frac{n}{2})^{2} $ + $ (n + 1)^{3} $ =
    $ \frac{n^{4}}{4} $ +
    $ \frac{3n^{3}}{2} $ + $ \frac{13n^{2}}{4} $ + 3n + 1 . So therfore
    $ \forall_{n} \geq 1 $, $ \sum_{i = 1}^{n} i^{3} $ = $ ( \sum_{i = 1}^{n} i )^{2} $
    is true for all natural numbers for n greater than or equal to 1.
    $ \newline \newline $
    (b) $ \forall_{n} \geq 4$, $ 2^{n} < n!$
    $ \newline \newline $
    Basis: for n = 4 $ 2^{4} $ = 16 and 4! = 24
    $ \newline \newline $
    Inductive Hypothesis: $ \forall_{n} \geq 4$, $ 2^{n} < n!$ is true for n.
    $ \newline \newline $
    Inductive Step: $ 2^{n+1} $ = $ 2^{n} * 2 $. (n + 1)! = n! * (n + 1)
    Since we know from the inductive hypothesis that $ \forall_{n} \geq 4$, $ 2^{n} < n!$
    so for this to remain true the right side and the left would need to be multiplied by the same value
    or the right would need to be multiplied by a greater value. The left is multiplied by 1 and the right is multiplied by
    (n + 1). We know that n has to be 4 or greater so we know (n + 1) > 2 so therfore $ \forall_{n} \geq 4$, $ 2^{n} < n!$
    for all natural numbers n.
    $ \newline \newline $
    2. Refer to the definition of Full Binary Tree from the notes. For a Full Binary Tree T, we
    use n(T), h(T), i(T) and `(T) to refer to number of nodes, height, number of internal nodes
    (non-leaf nodes) and number of leaves respectively. Note that the height of a tree with single
    node is 1 (not zero). Using structural induction, prove the following:
    $ \newline \newline $
    (a) For every Full Binary Tree T, n(T) $ \geq $ h(T).
    $ \newline \newline $
    Basis: For a full binary tree T with a root node and two child leaf nodes 
    $ n(T_{*}) = 3 $, and $ h(T_{*}) = 2 $ so $ n(T_{*}) \geq  h(T_{*}) $ is true.
    $ \newline \newline $
    Inductive Hypothesis: Suppose that for complete binary trees T,  n(T) $ \geq $ h(T) is true.
    $ \newline \newline $
    Recursive: Suppose we have two sub trees $ T_{1} , T_{2} $, these threes are identical full binary trees and when we combine them together with an addictional node 
    $ T_{1} + T_{2} N_{1} $ = T, $ n(T_{1,2})  \geq  h(T_{1,2}) $. First we will compare $ n(T_{1}) + n(T_{2}) + 1 \geq  h(T_{1,2}) + 1 $. $ n(T_{1}) + n(T_{2}) + 1 $ = n(T)
    and h(T) = $ h(T_{1,2}) $ + 1. We can reduce it down to $ 2n(T_{1}) + 1 \geq  h(T_{1}) + 1 $ .... $ 2n(T_{1}) \geq  h(T_{1}) $. By the IH we know that $ n(T_{1}) \geq  h(T_{1}) $
    is true so $ 2n(T_{1}) \geq  h(T_{1}) $ must also be true so therfore n(T) $ \geq $ h(T) is true for all full binary trees T.
    $ \newline \newline $
    (b) For every Full Binary Tree T, i(T) $ \ge $ h(T) - 1
    $ \newline \newline $
    Basis: For a full binary tree $ T_{*} $ with a single root node and two child leaf nodes. 
    $ i(T_{*}) $ = 1 and $ h(T_{*}) - 2 $ = 1 so therfore $ i(T_{*})  \ge  h(T_{*}) - 1 $
    $ \newline \newline $
    Inductive Hypothesis: Assume that i(T) $ \ge $ h(T) - 1 is true.
    $ \newline \newline $
    Recursive: We will take two idential full binary trees $ T_{1,2} $. If we  
    combined them with a single node to connect them into a new full binary tree T
    then to find all the internal nodes of T we would do $ i(T_{1}) + i(T_{2}) + 1 \ge h(T) - 1 $. 
    The height of T is one more than $ T_{1} $ so $ i(T_{1}) + i(T_{2}) + 1 \ge h(T_{1}) $. This can also 
    be reduced down to $ i(2T_{1}) + 1 \ge h(T_{1}) - 1 $. By the IH $ i(T_{1}) + 1 \ge h(T_{1}) - 1 $ 
    so $ i(2T_{1}) + 1 \ge h(T_{1}) - 1 $ is also true and therfore i(T) $ \ge $ h(T) - 1 is true
    for all complete binary trees T.
    $ \newline \newline $
    (c) For every Full binary Tree T, $ l(T) = (n(T) + 1)/2 $
    $ \newline \newline $
    Basis: for a full binary tree $T^{2} $ with a single root node and two children has 2 leaf nodes and 3 alltogether.
    so (3 + 1)/2 = 2, so therefore $ l(T^{3}) = (n(T^{3}) + 1)/2 $ is true for the basis.
    $ \newline \newline $
    Inductive Hypothesis: We assume $ l(T) = (n(T) + 1)/2 $ is true.
    $ \newline \newline $
    Recursive Step: We will take two idential full binary trees $ T_{1,2} $. If we  
    combined them with a single node to connect them into a new full binary tree T.
    By the IH $ l(T_{1,2}^{3}) = (n(T_{1,2}^{3}) + 1)/2 $. To find the number of leaves
    for T we can add these together so $ \frac{n(T_{1}) + 1}{2} $ + $ \frac{n(T_{2}) + 1}{2} $.
    We can simpllify this to because $ T_{1} $ = $ T_{2} $ , $ n(T_{1}) + 1 $ = l(T). 
    Now me must prove this is correct. We know that $ l(T) = (n(T) + 1)/2 $ so we will 
    expand n(T) in terms of $ T_{1} $ so $ l(T) = (2n(T_{1}) + 1 + 1)/2 $ ... l(T) = $ n(T_{1}) + 1 $.
    So because we were able to prove for T with two sub trees therefore $ l(T) = (n(T) + 1)/2 $ is true for 
    all full binary trees T.
    $ \newline \newline $
    3. Let a be an array of size n indexed by 0, 1, ... n − 1. Consider the following code (inner loop
    of Bubble Sort)
    $ \newline \newline $
    for i in the range [0, n-2]
    $ \newline $
        if (a[i] $>$ a[i+1])
        $ \newline $
            swap (a[i], a[i+1]); //swap the values of a[i] and a[i+1]

    $ \newline \newline $
    Show the following using induction :“ At the start of ith iteration of the loop, a[i] is the
    maximum among a[0], a[1], ... a[i].
    $ \newline \newline $
    Basis: for a[0] there are no values index in a below 0 so therfore a[i] where i = 0 
    the start of the iteration of i is the maximum.
    $ \newline \newline $
    inductive Hypothesis: Assume that a[i] is the greatest max value during the iteration 
    and that m(a[i]) is either true or false if a[i] is greatest during the ith iteration.
    $ \newline \newline $
    Inductive Step: By the IH assume that m(a[i]) is true so. If during the ith interation 
    a[i] $>$ a[i+1] then a[i+1] = a[i] in that case for m(a[i+1]) a[i+1] $>$ a[i] and m(a[i]) is true by
    the inductive hypothesis so therefore for that that m(a[i+1]) is true. The other
    case is that a[i+1] $>$ a[i] during the ith iteration. In that case its the same
    where m(a[i]) is already true so starting at the i+1 iteration we already know a[i]
    is greater than anything before it and the value after a[i], is greater a[i+1] $>$ a[i]
    so therefore m(a[i]) or that a[i] is the greatest max for a[0],a[1],...a[i] for all
    natural numbers i where $ 0 \leq i \leq n-1 $.
    $ \newline \newline $
    4. Let a be an array of size n indexed by 0, 1, ... n − 1. Now consider Bubble sort pseudo code.
    $ \newline \newline $
    for j in the range [0, n-1]
    $ \newline \newline $
    for i in the range [0, n-j-2]
    if a[i] > a[i+1]
    $ \newline \newline $
    swap(a[i], a[i+1])
    $ \newline \newline $
    Use mathematical induction to show: At the start of jth iteration of outer loop the following
    conditions hold:
    $ \newline \newline $
    a[n - j] $ \leq $ a[n - j + 1] $ \leq $ a[n - 1] 
    $ \newline \newline $
    a[n - j], a[n - j + 1],... a[n - 1] are the j largest elements of the array
    $ \newline \newline $
    Basis: Using induction on j if j = 2 then $ a[n - 2] \leq a[n - 1] \leq a[n - 1] $
    obviously $ a[n - 1] \leq a[n - 1] $. For arrays greater than 1 element.
    For the prior iteration where j = 1 then the inner for loop would have went from
    0 to n-1. For the iner loop when i = n - 2 if a[n - 2] $>$ a[n - 1] then 
    it would swap and put the greater value in a[n - 1] otherwise the greater value would
    already be there. So a[n - 2] $<$ a[n - 1]. 
    $ \newline \newline $
    Inductive Hypothesis: Assume that a[n - j] $ \leq $ a[n - j + 1] $ \leq $ a[n - 1] is true.
    $ \newline \newline $
    Inductive Step: Now we must prove for j + 1 so
    a[n - j + 1] $ \leq $ a[n - j + 2] $ \leq $ a[n - 1]. We know that j works
    so during the jth iteration 
    
\end{document}
