\documentclass[11pt]{article}
    %	options include 12pt or 11pt or 10pt
    %	classes include article, report, book, letter, thesis

    \usepackage{amsmath}
    \usepackage{array}
    \setlength\extrarowheight{2pt}
    \usepackage{graphicx}
    \graphicspath{ {/home/shanedrafahl/coms331/hw0} }

    \title{HW1}
    \author{Shane Drafahl}
    \date{30 August,2017}

    \begin{document}
    \maketitle

    1. Use mathematical induction to prove the following statements
    $ \newline \newline $
    (a)
    $ \newline $
    $ \forall_{n} \geq 1 $, $ \sum_{i = 1}^{n} i^{3} $ = $ ( \sum_{i = 1}^{n} i )^{2} $
    $ \newline \newline $
    $ \newline $
    lemma: $ \sum_{i = 1}^{n} i $ = $ \frac{n^{2}}{2} $ + $ \frac{n}{2} $ where $ n \in N $
    $ \newline $
    Base Case: $ \sum_{i = 1}^{1} i $ = 1 = $ \frac{1^{2}}{2} $ + $ \frac{1}{2} $
    $ \newline $
    Inductive Hypothesis: $ \sum_{i = 1}^{n} i $ = $ \frac{n^{2}}{2} $ + $ \frac{n}{2} $
    $ \newline $
    Inductive Steps:
    $ \newline $
    Using induction on n
    $ \newline $
    $ \frac{(n + 1)^{2}}{2} + \frac{n + 1}{2} $ = $ \frac{n^{2}}{2} $ + $ \frac{3n}{2} $ + 1 =
    $ \frac{n^{2}}{2} + \frac{n}{2} $ + $ (n + 1) $
    $ \newline \newline $
    So therefore the lemma is true for all natural numbers n greater than 1.
    $ \newline \newline $
    With the lemma we can solve the proof easier
    $ \newline \newline $
    Base Case: For n = 1 would be $ \sum_{i = 1}^{1} i^{3} $ = $ 1^{3} $ = 1.
    $ ( \sum_{i = 1}^{1} i )^{2} $ = 1.
    $ \newline \newline $
    Inductive Hypothesis: From the base n is true for
    $ \forall_{n} \geq 1 $, $ \sum_{i = 1}^{n} i^{3} $ = $ ( \sum_{i = 1}^{n} i )^{2} $
    $ \newline \newline $
    Inductive Step: Using lemma we can re-write this to
    $ (\frac{n^{2}}{2}  +  \frac{n}{2})^{2} $ = $ \sum_{i = 1}^{n} i^{3} $
    We will first add n+1, $( \frac{(n + 1)^{2}}{2}  +  \frac{(n + 1)}{2})^{2} $ = $ \frac{n^{4}}{4} $ +
    $ \frac{3n^{3}}{2} $ + $ \frac{13n^{2}}{4} $ + 3n + 1 .
    Now we will do $ ( \frac{n^{2}}{2}  +  \frac{n}{2})^{2} $ + $ (n + 1)^{3} $ =
    $ \frac{n^{4}}{4} $ +
    $ \frac{3n^{3}}{2} $ + $ \frac{13n^{2}}{4} $ + 3n + 1 . So therfore
    $ \forall_{n} \geq 1 $, $ \sum_{i = 1}^{n} i^{3} $ = $ ( \sum_{i = 1}^{n} i )^{2} $
    is true for all natural numbers for n greater than or equal to 1.
    $ \newline \newline $
    (b) $ \forall_{n} \geq 4$, $ 2^{n} < n!$
    $ \newline \newline $
    Basis: for n = 4 $ 2^{4} $ = 16 and 4! = 24
    $ \newline \newline $
    Inductive Hypothesis: $ \forall_{n} \geq 4$, $ 2^{n} < n!$ is true for n.
    $ \newline \newline $
    Inductive Step: $ 2^{n+1} $ = $ 2^{n} * 2 $. (n + 1)! = n! * (n + 1)
    Since we know from the inductive hypothesis that $ \forall_{n} \geq 4$, $ 2^{n} < n!$
    so for this to remain true the right side and the left would need to be multiplied by the same value
    or the right would need to be multiplied by a greater value. The left is multiplied by 1 and the right is multiplied by
    (n + 1). We know that n has to be 4 or greater so we know (n + 1) > 2 so therfore $ \forall_{n} \geq 4$, $ 2^{n} < n!$
    for all natural numbers n.
    $ \newline \newline $
    2. Refer to the definition of Full Binary Tree from the notes. For a Full Binary Tree T, we
    use n(T), h(T), i(T) and `(T) to refer to number of nodes, height, number of internal nodes
    (non-leaf nodes) and number of leaves respectively. Note that the height of a tree with single
    node is 1 (not zero). Using structural induction, prove the following:
    $ \newline \newline $
    (a) For every Full Binary Tree T, n(T) $ \geq $ h(T).
    $ \newline \newline $
    Basis: For a full binary tree T with a root node and two child leaf nodes 
    $ n(T_{*}) = 3 $, and $ h(T_{*}) = 2 $ so $ n(T_{*}) \geq  h(T_{*}) $ is true.
    $ \newline \newline $
    Inductive Hypothesis: Suppose that for complete binary trees T,  n(T) $ \geq $ h(T) is true and n(T) = $ 2^{h(T)} - 1 $.
    $ \newline \newline $
    Recursive: Suppose we have two sub trees $ T_{1} , T_{2} $ that when combined with a node to connect them 
    $ n_{0} + T_{1} + T_{2}$ = T. So to find the total number of nodes for n(T) = $n(T_{1}) +  n(T_{2}) + 1$ which because $ T_{1} = T_{2} $
    n(T) = $n(T_{1}) +  n(T_{2})$ can be reduced to n(T) = $2n(T_{1}) + 1$ or by using the IH $ 2^{h(T_{1}) + 1} $. We know that the height of
    T will increase by 1 so $ h(T) = h(T_{1}) + 1 $. So $  2^{h(T_{1}) + 1}  \geq  h(T)  $ or $  2^{h(T)}  \geq  h(T)  $ which is clearly true so for
    all values of h(T) so therefore n(T) $ \geq $ h(T) is true for all complete binary trees T. 
    $ \newline \newline $
    (b) For every Full Binary Tree T, i(T) $ \ge $ h(T) - 1
    $ \newline \newline $

    (c) For every Full binary Tree T, $ l(T) = (n(T) + 1)/2 $
    $ \newline \newline $
    Basis: for a full binary tree $T^{2} $ with a single root node and two children has 2 leaf nodes and 3 alltogether.
    so (3 + 1)/2 = 2, so therefore $ l(T^{3}) = (n(T^{3}) + 1)/2 $ is true for the basis.
    $ \newline \newline $
    Inductive Hypothesis: We assume $ l(T) = (n(T) + 1)/2 $ is true and that the n(T) = $ 2^{h(T)} - 1 $.
    $ \newline \newline $
    Recursive Step: Suppose we have $ T_{1} $ and $ T_{2} $ full binary trees where 
    $ l(T_{1, 2}) = (n(T_{1, 2}) + 1)/2 $ . If we combined $ T_{1} $ and $ T_{2} $ into a full tree T then 
    the leaves of the new tree would be
    $ \frac{(n(T^{1}) + 1)}{2} $ + $ \frac{(n(T^{2}) + 1)}{2} $.
    Since $ T_{1} $ and $ T_{2} $ are equivilant we can say $ n(T_{1}) + 1 $.
    With our IH $ n(T_{1}) + 1 $ = $ 2^{h(T_{1}) - 1} $. also from the IH
    we can say that IH of $ l(T) = 2^{h(T) - 1} $. Because $ h(T) = h(T_{1}) + 1 $ 
    $ l(T) = 2^{h(T) - 1} =  2^{h(T_{1})} $. So therefore $ l(T^{3}) = (n(T^{3}) + 1)/2 $ is true for all full binary trees T.

    \end{document}
