\documentclass[11pt]{article}
    %	options include 12pt or 11pt or 10pt
    %	classes include article, report, book, letter, thesis
    
    \usepackage{amsmath}
    \usepackage{array}
    \setlength\extrarowheight{2pt}
    \usepackage{graphicx}
    \graphicspath{ {/home/shanedrafahl/coms331/hw0} }
    
    \title{HW1}
    \author{Shane Drafahl}
    \date{30 August,2017}
    
    \begin{document}
    \maketitle
    
    1. Use mathematical induction to prove the following statements
    $ \newline \newline $
    (a)
    $ \newline $
    $ \forall_{n} \geq 1 $, $ \sum_{i = 1}^{n} i^{3} $ = $ ( \sum_{i = 1}^{n} i )^{2} $
    $ \newline \newline $
    $ \newline $
    lemma: $ \sum_{i = 1}^{n} i $ = $ \frac{n^{2}}{2} $ + $ \frac{n}{2} $ where $ n \in \mathbb{N} $
    $ \newline $
    Base Case: $ \sum_{i = 1}^{1} i $ = 1 = $ \frac{1^{2}}{2} $ + $ \frac{1}{2} $ 
    $ \newline $
    Inductive Hypothesis: $ \sum_{i = 1}^{n} i $ = $ \frac{n^{2}}{2} $ + $ \frac{n}{2} $ 
    $ \newline $
    Inductive Steps: 
    $ \newline $
    Using induction on n
    $ \newline $
    $ \frac{(n + 1)^{2}}{2} + \frac{n + 1}{2} $ = $ \frac{n^{2}}{2} $ + $ \frac{3n}{2} $ + 1 =
    $ \frac{n^{2}}{2} + \frac{n}{2} $ + $ (n + 1) $
    $ \newline \newline $
    So therefore the lemma is true for all natural numbers n greater than 1. 
    $ \newline \newline $    
    With the lemma we can solve the proof easier
    $ \newline \newline $
    Base Case: For n = 1 would be $ \sum_{i = 1}^{1} i^{3} $ = $ 1^{3} $ = 1.
    $ ( \sum_{i = 1}^{1} i )^{2} $ = 1. 
    $ \newline \newline $
    Inductive Hypothesis: From the base n is true for 
    $ \forall_{n} \geq 1 $, $ \sum_{i = 1}^{n} i^{3} $ = $ ( \sum_{i = 1}^{n} i )^{2} $
    $ \newline \newline $
    Inductive Step: Using lemma we can re-write this to 
    $ (\frac{n^{2}}{2}  +  \frac{n}{2})^{2} $ = $ \sum_{i = 1}^{n} i^{3} $
    We will first add n+1, $( \frac{(n + 1)^{2}}{2}  +  \frac{(n + 1)}{2})^{2} $ = $ \frac{n^{4}}{4} $ +
    $ \frac{3n^{3}}{2} $ + $ \frac{13n^{2}}{4} $ + 3n + 1 .
    Now we will do $ ( \frac{n^{2}}{2}  +  \frac{n}{2})^{2} $ + $ (n + 1)^{3} $ = 
    $ \frac{n^{4}}{4} $ +
    $ \frac{3n^{3}}{2} $ + $ \frac{13n^{2}}{4} $ + 3n + 1 . So therfore 
    $ \forall_{n} \geq 1 $, $ \sum_{i = 1}^{n} i^{3} $ = $ ( \sum_{i = 1}^{n} i )^{2} $
    is true for all natural numbers for n greater than or equal to 1.
    $ \newline \newline $
    (b) $ \forall_{n} \geq 4$, $ 2^{n} < n!$
    $ \newline \newline $
    Basis: for n = 4 $ 2^{4} $ = 16 and 4! = 24
    $ \newline \newline $
    Inductive Hypothesis: $ \forall_{n} \geq 4$, $ 2^{n} < n!$ is true for n.
    $ \newline \newline $
    Inductive Step: $ 2^{n+1} $ = $ 2^{n} * 2 $. (n + 1)! = n! * (n + 1)
    Since we know from the inductive hypothesis that $ \forall_{n} \geq 4$, $ 2^{n} < n!$
    so for this to remain true the right side and the left would need to be multiplied by the same value
    or the right would need to be multiplied by a greater value. The left is multiplied by 1 and the right is multiplied by
    (n + 1). We know that n has to be 4 or greater so we know (n + 1) > 2 so therfore $ \forall_{n} \geq 4$, $ 2^{n} < n!$
    for all natural numbers n.
    $ \newline \newline $
    2. Refer to the definition of Full Binary Tree from the notes. For a Full Binary Tree T, we
    use n(T), h(T), i(T) and `(T) to refer to number of nodes, height, number of internal nodes
    (non-leaf nodes) and number of leaves respectively. Note that the height of a tree with single
    node is 1 (not zero). Using structural induction, prove the following:
    $ \newline \newline $
    (a) For every Full Binary Tree T, n(T) $ \geq $ h(T).
    $ \newline \newline $
    Basis: For a tree T with a single root node has a height 1. It has a single node and 1 = 1 so
    for n(T) $ \geq $ h(T) is true.
    $ \newline \newline $
    Inductive Hypothesis: n(T) $ \geq $ h(T) is true is true for single root node tree T.
    $ \newline \newline $
    Inductive Step: Suppose you have a tree T that n(T) $ \geq $ h(T) is true. If we want to 
    add more nodes to the tree and for it to also be a full binary tree we must add two nodes per leaf node. The root
    node doesnt have a parent node so its child nodes are the first nodes to branch into two different child nodes.
    So starting from the roots direct child node one layer difference thus $ (h(T) - 1)^{2} $ equals the number of 
    leaf nodes. So increase the height of the tree by 1 would require two nodes for every leaf node thus for the tree with 
    increased height $ T_{a} $. So h($T_{a}$) = h(T) + 1 and n($ T_{a} $) = n(T) * 2. So n(T) * 2 $ \geq $  h(T) + 1 is true
    by the inductive hypothesis because we already know n(T) $ \geq $ h(T) and n(T) is being multiplied by 2 and  h(T) is
    only added by 1. So the left side which is bigger is being increased more than the right so therfore n(T) $ \geq $ h(T)
    is true for all complete binary search trees T.

    \end{document}