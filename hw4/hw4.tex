\documentclass[11pt]{article}
    %	options include 12pt or 11pt or 10pt
    %	classes include article, report, book, letter, thesis

    \usepackage{amsmath}
    \usepackage{array}
    \setlength\extrarowheight{2pt}
    \usepackage{graphicx}
    \usepackage{epstopdf}
    \usepackage{graphics}
    \graphicspath{ {/home/shanedrafahl/coms331/hw0} }
    
    \title{HW5}
    \author{Shane Drafahl}
    \date{31 October,2017}

    \begin{document}
    \maketitle

    1. This algorithm goes over every node in the graph and checks the neighbors of the neighbors
     and populates a adjacency list for nodes of distance 2 from each other. After this the adjacency
     list is then set to what the new adjacency list is used.

     $ \newline $
    
    \begin{verbatim}
       
        // Given a graph G createG2 converts G to G2 graph
        function createG2(G) {
            foreachNode(G as N) {
                foreachNeighbor(N as N1) {
                    foreachNeighbor(N1 as N2) {
                        append(N.N2, N2);
                    }
                }
            }
            foreachNode(G as N) {
                N.N = N.N2;
            }
        }

        Node {
            N[] // adjacency list
            N2[] // adjacency list for G squared 
        }

        Graph {
            N[] // List of nodes 
        }

    \end{verbatim}

    $ \newline $

    This algorithm runs in $ O(nm) $ because the first for loop goes over n vertices and the second for loop that
    goes over every vertice. That means the two for loops is just $ O(n + m) $. There is a fourth for loop 
    though that also goes over vertices so it is $ O(m(n + m)) = O(mn + m^{2}) $. $ m * n $ is the biggest factor 
    because m at most can be $ m = n^{2} $ so therefore it is $ O(mn) $.

    $ \newline $

    \begin{verbatim}
        
         // Given a adjacency matrix where (x, y) and x is the domain and the y is the co-domain if
         // there is a function between different vertices to other vertices.
         function createG2(M) { // suppose that M is a adjacency matrix.
                M2 // new Matrix of equal size and width of M
                x = 0
                for(x in range(M.height)) {
                    y = 0
                    for(y in range(M.width)) {
                        if(M[x][y] == 1) {
                            a = 0
                            for(a in range(M.width)) {
                                if(M[y][a] == 1) {
                                    M2[x][a] = 1
                                }
                            }
                        }
                    }
                }
                return M2
         }
 
     \end{verbatim}

    $ \newline $

    This algorithm goes over every index of the matrix which is $ O(n^{2}) $ where $ n $ 
    is the number of vertices. If it finds a connection it has another for loop an then updates the initial row.
    Worst case scenario this is a fully connected graph so $ O(n^{3}) $ best case the graph has no edges or connections
    it would be $ O(n^{2}) $. 

    $ \newline $

    2. For this problem I will use Kosaraju's algorithm to see if there is a vertice with equal number of nodes behind it
    as in front of it.

    $ \newline $

    \begin{verbatim}
        
         
         function hasCenter(G) { // suppose G is the Grahp
                ////////////////////////////////////////// Kosaraju algorithm
                L // list of strongly connected nodes with only a single node
                S = new Stack // new stack
                forEachNode(G as N) {
                    if(!N.visited) {
                        N.visited = true
                        dfs(N, S)
                    }
                }
                while(!isEmpty(S)) {
                    Node = pop(S)
                    if(Node.Root == null) {
                        dfsStronglyConnected(Node, Node, L)
                    }
                }
                ////////////////////////////////////////////
                forEach(L as STN) { // each element in L as STN or strongly connect node
                    size = countRightNodes(STN) - 1
                    sizeOfGraph = size(G.L)
                    sizeOfGraph = (sizeOfGraph - 1)/2
                    if(sizeOfGraph == size) {
                        return STN
                    }
                }
                return false
         }

         // recursive algorithm that creates the strongly connected vertices
         // Root is a Node that is chosen that will represent the strongly connected graph
         function dfsStronglyConnected(Node, Root, L) { 
            Node.Root = Root
            Root.quant++
            foreachNeighbor(Node as N) {
                dfsStronglyConnected(N, Root, L)
            }
            if(Root.quant == 1) {
                append(L, Node) // adds the Node to the list L
            }
         }

         function countRightNodes(N) {
             N.visited = false
             if(hashNeighbors(N)) {
                 foreachNeighbor(N as N1) {
                     if(N1.visit = true) {
                        return 1 + countRightNodes(N1)
                     }
                 }
             }
         }

         // N is a node
         // S is a stack
         function dfs(N, S) {
             if(hashNeighbors(N)) {
                 foreachNeighbor(N as N1) {
                     dfs(N1, S)
                 }
             } else {
                 push(S, N) // pushes N onto stack S
             }
         }  

         Node {
            Root // root of the node
            quant = 0 // number of nodes it represents
            visited = false
            Neighbors[] // adjacent neighbors
         }

         Graph {
             L // list of nodes
         }
 
     \end{verbatim}

     $ \newline $

     The algorithm above first uses Kosaraju's algorithm to find all the strongly connected
     components in the graph and connect each one to a root that contains the number of nodes
     in each strongly connected component. Once this is done I keep track of the strongly connected
     nodes with only a single node because $ v $ in the algorithm will be a single strongly connected
     component with a single node because the set of From and To cannot overlap. I then itterate over all 
     strongly connected components with single nodes and using DFS find the number of nodes to the right. 
     If the graph has a center then there should be exactly k nodes to the right of v. So Once I find
     the total number of nodes to the right I take the total number of nodes for the whole graph 
     subtract 1 and divided by 2 to get k. If the number of nodes to the right of v equal k then
     I know the graph has a center. 

     $ \newline $

     Kosaraju's algorithm runs in linear time so $ O(n + E) $. The forloop after the algorithm runs through
     every strongly connected component. Worst case every node could be strongly connected so it would 
     be essentialy a linked list. In this case worst case be k iterations. So it would be 
     $ O(n + E + \frac{n - 1}{2}) $ or $ O(n + E) $.

     $ \newline $

     3. Is there a vertex $ u \in V $ such if we perform DFS on G starting at $ u $ , the vertex $ v $
     will be a leaf node in the resulting DFS tree?

     $ \newline $

     No, a leaf node in the recursion tree for DFS can only happen if a node or a vertice is 
     reached without calling any adjacent vertices. A bridge node cannot be the leaf node or else 
     one side of the graph would be unvisited.

     $ \newline $

     Because this algorithm must be in $ O(n + m) $ we know it must be DFS based. From the previous question we
     know it cannot be a leaf node so we simply just need to do a DFS search and stop once we reach a leaf node and return
     it.

     $ \newline $

     \begin{verbatim}
        
        function NonBridgeDFS(G) {
            return search(G.adj[0])
        }

        // N is a vertice or a node
        function search(N) {
            N.visited = true
            leaf = true
            foreachNeighbor(N as N1) {
                if(N1.visited == false) {
                    leaf = false
                    return search(N1)
                }
            }
            if(leaf) {
                return N
            }
        }
        
        Node {
            visited = false
        }

 
     \end{verbatim}

     This algorithm is correct because we know that a leaf node cannot be a bridge node. A leaf node is one where 
     when search is used on the node it cannot find any more adjacent Nodes that have not been visited before. The graph
     is connected so we can pick any arbitrary node or vertice. 

     $ \newline $

     This algorithm is $ O(n + m) $ where n is the number of vertices and m is the number of edges. The recursive call visit 
     visits every node in the graph untill it reaches a leaf node so it is $ O(n) $. For each vertice the foreachNeighbor() 
     goes over every edge. So every edge and every node is itterated at most. Therefore its runtime is $ O(n + m) $.



    \end{document}

