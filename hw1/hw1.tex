\documentclass[11pt]{article}
    %	options include 12pt or 11pt or 10pt
    %	classes include article, report, book, letter, thesis

    \usepackage{amsmath}
    \usepackage{array}
    \setlength\extrarowheight{2pt}
    \usepackage{graphicx}
    \graphicspath{ {/home/shanedrafahl/coms331/hw0} }

    \title{HW2}
    \author{Shane Drafahl}
    \date{30 August,2017}

    \begin{document}
    \maketitle

    1. Prove or disprove the following
    $ \newline \newline $
    (a) $ 5n^{2} - 2n + 26 \in O(n^{2}) $
    $ \newline \newline $
    We will prove this with the def of big oh.
    The def of $ O(n^{2}) $ is there exists positive constants c and $ n_{0} $ such
    that 0 $ \leq $ f(n) $ \leq $ c * g(n) for all $ n_{0} \leq n $. In this case 
    f(n) = $ 5n^{2} - 2n + 26 $ and g(n) = $ n^{2} $. We can divide both sides by
    $ n^{2} $ and we can go from $ 0 \leq 5n^{2} - 2n + 26 \leq c * n^{2} $ to
    $ 0 \leq \frac{24}{n} + 5 \leq c $. $ \frac{24}{n = 1} + 5 $ = 29 and if
    $ f_{1}(n) = \frac{24}{n} + 5 $ then $f_{1}(n + 1) \leq f_{1}(n) $ because as 
    natural number n increases it increases the denominator. C could be 29
    or greater and $ 0 \leq 5n^{2} - 2n + 26 \leq c * n^{2} $ would be true so therefore
    $ 5n^{2} - 2n + 26 \in O(n^{2}) $ because the property is true.
    $ \newline \newline $
    (b) $ \forall_{a} \geq 1 $ : $ a^{n} \in O(n!) $
    $ \newline \newline $
    We will prove this with def of big oh.
    So the given statement is equivalent to  $ 0 \leq a^{n} \leq c * n! $. Using induction
    we can prove it.
    $ \newline \newline $
    Basis: Starting at n = 1 because the performance of an algorithm with n = 0 is irrelevent. 
    $ 0 \ leq a \leq c $ is true because for any a c can be a constant of c  = (a + 1).
    Inductive Hypothesis: Suppoose $ 0 \leq a^{n} \leq c * n! $ is true.
    $ \newline \newline $
    Inductive Step:
    We need to prove $ 0 \leq a^{n + 1} \leq c * n! * (n + 1) $. $ a^{n + 1} $ increases by
    some a multiplied by $ a * a^{n} $ from the IH. While the right side $ c * n! * (n + 1) $ from the
    IH is multiplied by (n + 1) for $ c * n! * (n + 1) $. In this case of a, n increases meaning at some point
    it will increase by more when a $<$ n. So the right side is increasing at a faster rate
    then the left side of the comparison. This means that there is a point where $ n! \leq a^{n} $ for some a 
    for a given range of n. We can just say $ c = a^{n} = n! $ for the n where they equal. So for 
    $ 0 \leq a^{n + 1} \leq c * n! * (n + 1) $ if $ a^{n} > n! $ then the constant c multiplied by 
    $ n! * c $ will be greater than or equal to $ a^{n} $ because c is equal to the the value 
    at which n! overtakes $ a^{n} $ so if for n! n is beyond the point where n! overtakes than it will already
    overtake and be a greater value. The other case is that $ a^{n} \leq n! $ for some n then it wont matter what
    c is because n! will be increasing at a greater rate. So $ 0 \leq a^{n + 1} \leq c * n! * (n + 1) $. $ a^{n + 1} $
    is true and therefore $ \forall_{a} \geq 1 $ : $ a^{n} \in O(n!) $ is true.
    $ \newline \newline $
    (c) $ \forall_{a} \geq $ : $ 2^{n + a} \in O(2^{n}) $
    $ \newline \newline $
    To prove for big oh We must prove for $ 0 \leq 2^{n + a} \leq c * 2^{n} $. 
    We can reduce this to $ 0 \leq 2^{n}2^{a} \leq c * 2^{n} $. Because for whatever a is
    we can say that c = $ 2^{a} $ for this value of a so $ 0 \leq 2^{n}2^{a} \leq 2^{a}2^{n} $.
    So obviously this is true You cannot produce a negative number from the exponents either.
    So therefore $ \forall_{a} \geq $ : $ 2^{n + a} \in O(2^{n}) $ is true for all a and n.
    $ \newline \newline $
    (d) $ \forall_{a} > 1 $ : $ (f(n) \in O(log_{2}n)) => (f(n) \in O(log_{a}n)) $
    $ \newline \newline $
    To prove that this is wrong I will use proof of contradiction. 
    Assuming that $ \forall_{a} > 1 $ : $ (f(n) \in O(log_{2}n)) => (f(n) \in O(log_{a}n)) $
    Then either $ f(n) \leq log_{a}(n) \leq log_{2}(n) $ or 
    $ f(n) \leq log_{2}(n) \leq log_{a}(n) $. 
    For $ f(n) \leq log_{2}(n) \leq log_{a}(n) $ a could equal a number greater than 2 and that would
    be a contradiction $ log_{2}(n) \leq log_{3}(n) $ if n is greater than 1.
    Otherwise $ f(n) \leq log_{1}(n) \leq log_{2}(n) $. For this case if f(n) = $ log_{3}(n) $ 
    then for $ log_{a}(n) $, a would have to be between 3 and 2 and a contradiction because 
    it cannot be all values. So therefore $ \forall_{a} > 1 $ : $ (f(n) \in O(log_{2}n)) => (f(n) \in O(log_{a}n)) $
    is not true.
    $ \newline \newline $
    (e) $ 2^{n} \in O(n^{log^{2}n}) $
    $ \newline \newline $
    To disprove this we will use proof of contradiction we will assume that $ 2^{n} \in O(n^{log^{2}n}) $
    is true so $ 0 \leq 2^{n} \leq c * n^{log(n)^{2}} $. If n = 5 then $ 2^{5} $ = 32, and $ 5^{log(5)^{2}} $ = 2.19527...
    So C would need to be 14.5767 or greater for this case. Since we assume it is true we also 
    assume that there is some C for that multiplied by $ n^{log(n)^{2}} $ that would be greater than
    $ 2^{n} $ for all values of n. We will assume that $ 0 \leq 2^{n} \leq c_{1} * n^{log(n)^{2}} $ is
    true where $ c_{1} $ would not need to be increased because its true for all values of n. If we 
    increase n by 1 then $ 2^{n + 1} = 2^{n} * 2 $ would increase 2 multiplied by $ 2^{n} $.
    On the right side $ (n + 1)^{log(n + 1)^{2}} $. For an increase of 1 for log(n) n would 
    have to be increase by log(n * 10) so log(n + 1) would increase less than 1 from log(n). So 
    $ (n + 1)^{log(n + 1)^{2}} $ would not double in size meaning that C would have to increase but
    by contradiction because we assumed that C was the greatest value it needed to be 
    $ 2^{n} \in O(n^{log^{2}n}) $ is not true.
    $ \newline \newline $
    (f) $ 2^{2^{n + 1}} \in O(2^{2^{n}}) $
    $ \newline \newline $
    To disprove this $ 0 \leq 2^{2^{n + 1}} \leq 2^{2^{n}} * C $ we will use proof by contradiction and
    assume this is true and that C is as larg as it needs to be. If we increase n by 1 we get 
    $ 0 \leq 2^{2^{n} * 4} \leq 2^{2^{n} * 2} * C $. $ 2^{2^{n} * 4} $ increases by more than
    $ 2^{2^{n} * 2} * C $ so C would need to be increased but this is a contradiction so therefore 
    by proof of contradiction $ 2^{2^{n + 1}} \in O(2^{2^{n}}) $ is not true.




    \end{document}
